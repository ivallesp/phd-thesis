% !TeX spellcheck = es_ES
% !TEX root = ../thesis.tex

\chapter*{Resumen}

\addcontentsline{toc}{chapter}{Resumen}

Los algoritmos de aprendizaje profundo representan el estado de la cuestión en lo que a aprendizaje automático se refiere. Muchas de sus aplicaciones requieren una gran cantidad de recursos computacionales, la cual limita su uso a dispositivos de alto rendimiento. El objetivo principal de esta tesis es estudiar métodos y algoritmos que permitan abordar problemas de aprendizaje profundo cuando se tienen recursos computacionales limitados. Este trabajo también tiene como objetivo presentar aplicaciones de aprendizaje profundo en la industria.

La primera contribución consiste en una nueva función de activación para redes de aprendizaje profundo: la función \textit{módulo}. A partir de los experimentos realizados se observa que la función de activación propuesta logra resultados superiores en tareas de visión artificial en comparación con las alternativas más avanzadas.

En segundo lugar, se presenta un nuevo método para combinar modelos pre-entrenados usando técnicas de destilación de conocimiento. Los resultados de este capítulo muestran el uso de las técnicas propuestas permite aumentar significativamente el desempeño de los modelos pre-entrenados más pequeños. Esto proporciona mejoras computacionales y de rendimiento.

La tercera aportación de esta tesis aborda el problema de la predicción de ventas en el campo de la logística. Se proponen dos sistemas de basados en dos técnicas diferentes de aprendizaje profundo (modelos de secuencia-a-secuencia y \textit{transformers}). De los resultados de este capítulo se concluye que es posible construir sistemas integrales para la predicción de ventas de múltiples productos individuales, en múltiples puntos de venta y en diferentes momentos en el tiempo, mediante el uso de un único modelo de aprendizaje automático. Los resultados del modelo propuesto superan significativamente a las alternativas encontradas en la literatura.

Finalmente, las dos últimas contribuciones pertenecen al campo de la tecnología del habla. El primero estudia cómo construir un sistema de reconocimiento de comandos de voz (\textit{Keyword Spotting}) utilizando una versión eficiente de una red neuronal convolucional. En este estudio, el sistema propuesto es capaz de superar el rendimiento de las alternativas encontrados en la literatura, en las tareas más complejas. El último estudio propone un modelo independiente de generación de habla capaz de sintetizar voz natural e inteligible usando miles de perfiles de voz distintos, generando habla expresiva con variaciones de prosodia significativas. El enfoque propuesto elimina la dependencia de los modelos anteriores en un sistema de voz de producción, lo que lo hace más eficiente en el tiempo de entrenamiento e inferencia.

% Keywords
%\vspace{1cm}

%\noindent \textit{Keywords}: machine learning, deep learning, tinyML, activation function, knowledge distillation, pre-trained models, sales forecasting, keyword spotting, text-to-speech.



\clearpage
